\documentclass[12pt]{article}

\usepackage{sbc-template}
\usepackage{graphicx,url}
\usepackage[utf8]{inputenc}
\usepackage[brazil]{babel}
\usepackage[latin1]{inputenc}  

     
\sloppy

\title{Projeto Final\\ Site SOFT MemoLife}

\author{César Eduardo de Souza, Guilherme Diel}


\address{Departamento de Ciência da Computação -- Universidade do Estado de Santa Catarina
  (UDESC)\\
  Caixa Postal 631 -- 89.219-710 -- Joinville -- SC -- Brazil
  \email{cesar.souza@edu.udesc.br, guilherme.diel1402@edu.udesc.br}
}

\begin{document} 

\maketitle

\begin{abstract}
  This report delivers the work \textbf{Final Project} of the discipline
  Software engineering (SOFT) consisting of the presentation of a solution
  of software, its requirements, duration estimates, UML diagram and
  unitary tests.
\end{abstract}
     
\begin{resumo} 
  Este relatório entrega o trabalho \textbf{Projeto Final} da disciplina 
  Engenharia de software (SOFT) que consiste na apresentação de uma solução
  de software, seus requisitos, estimativas de duração, diagrama UML e
  testes unitários.
\end{resumo}


\section{Enunciado do trabalho} % talvez não botar
No projeto final de SOFT você e sua equipe deverão selecionar um problema a ser solucionado por um software, e especificar o desenvolvimento de tal software. Os seguintes itens devem ser documentados na forma de um documento de texto para a entrega:
\subsection{Descrição do problema}
contendo o escopo do software e listagem dos stackeholders.
\subsection{Requisitos do software}
Que poderá ser na forma descritiva de requisitos funcionais e não funcionais, ou histórias de usuário ou protótipos do sistema.
\subsection{Estimativa de duração do projeto completo}
(Usando Cocomo ou outro método - é preciso mostrar e descrever os passos)
\subsection{Diagrama de Classes do Projeto UML}
contendo as principais classes do projeto, seus atributos e métodos. A estrutura das classes deve utilizar os padrões de projeto aprendidos sempre que for necessário. Entende-se por necessário o uso de padrões quando o problema relacionado ao padrão se aplique sobre o caso modelado.
\subsection{Testes Unitários}
a partir do Diagrama de Classes do Projeto, gere um "esqueleto" do código-fonte do projeto. Em seguida, elabore testes unitários para as classes do projeto. Caso não for possível 100\% de cobertura de testes das classes, foque nos métodos mais críticos do projeto justificando sua criticidade no texto. Crie um repositório na nuvem para colocar as classes produzidas e compartilhe o link com acesso ao mesmo no documento de texto. Este repositório deverá conter:
\begin{itemize}
  \item todos os códigos-fonte das classes do projeto
  \item todos os testes unitários produzidos
\end{itemize}
\subsection{Observação} 
Você poderá utilizar a linguagem de programação e framework de testes que lhe for mais conveniente
No dia da entrega, será programada uma defesa para cada trabalho em que a equipe irá demonstrar os testes elaborados
As equipes poderão ser compostas por no máximo 3 pessoas.

\section{Descrição do problema} \label{sec:firstpage}
O problema consiste num site escrito em HTML, CSS, Javascript e Ruby on Rails; 
visando permitir com que as pessoas se expressem, focando no compartilhamento 
de sentimentos.

\section{Requisitos}
\subsection{Funcionais}
\begin{itemize}
  \item Fazer posts
  \item Mostrar todos os posts do usuário
  \item Criar conta
  \item Fazer login 
\end{itemize}
\subsection{Não-funcionais}
\begin{itemize}
  \item A senha da conta deve conter entre 6 e 20 caracteres
  \item A publicação de posts deve ser imediata
\end{itemize}
\subsection{Histórias de usuário} % Não sei se precisa
  Entra no site, cria a conta, faz o login, clica para iniciar post, 
  preenche os campos do post, publica e logo em seguida vê o post na lista de posts.
\subsection{Protótipo de sistema}
  O protótipo do sistema está disponível em código fonte para ser 
  implementado em https://github.com/guilhermedd/site-SOFT 

\section{Estimativa de duração do projeto completo}
\subsection{Sobre o método Cocomo}
  O método COCOMO é um modelo utilizado para estimar o custo e o esforço envolvidos 
  no desenvolvimento de software. Foi proposto por Barry W. Boehm em 1981 e 
  tem sido amplamente utilizado na indústria de software para estimar recursos, 
  tempo e custos envolvidos em projetos de desenvolvimento de software.

\bibliographystyle{sbc}
\bibliography{sbc-template}

\end{document}
